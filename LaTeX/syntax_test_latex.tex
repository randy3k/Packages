% SYNTAX TEST "Packages/LaTeX/LaTeX.sublime-syntax"

% <- text.tex.latex

\documentclass[12pt]{article}
% ^ keyword.control.preamble.latex
%              ^ variable.parameter.latex
%                    ^ support.class.latex

\usepackage[args]{mypackage}
% ^ keyword.control.preamble.latex
%           ^ variable.parameter.latex
%                 ^ support.class.latex

% line comment
% <- comment.line.percentage.tex

\begin{document}
% ^ support.function.be.latex
%        ^ variable.parameter.function.latex

% SECTION COMMANDS

\part{name}
% <- meta.section.latex
% ^ support.function.section.latex
%     ^ entity.name.section.latex
\chapter{name}
% <- meta.section.latex
% ^ support.function.section.latex
%        ^ entity.name.section.latex
\section{name}
% <- meta.section.latex
% ^ support.function.section.latex
%        ^ entity.name.section.latex
\subsection{name}
% <- meta.section.latex
% ^ support.function.section.latex
%           ^ entity.name.section.latex
\subsubsection{name}
% <- meta.section.latex
% ^ support.function.section.latex
%              ^ entity.name.section.latex
\paragraph{name}
% <- meta.section.latex
% ^ support.function.section.latex
%          ^ entity.name.section.latex
\subparagraph{name}
% <- meta.section.latex
% ^ support.function.section.latex
%             ^ entity.name.section.latex


% MARKUP COMMANDS

\emph{text}
%     ^ markup.italic.emph.latex
\textbf{text}
%       ^ markup.bold.textbf.latex
\textit{text}
%       ^ markup.italic.textit.latex
\texttt{text}
%       ^ markup.raw.texttt.latex
\textsl{text}
%       ^ markup.italic.textsl.latex
\textbf{\textit{text}}
%               ^ markup.bold.textbf.latex markup.italic.textit.latex
\textit{\textbf{text}}
%               ^ markup.italic.textit.latex markup.bold.textbf.latex
\underline{text}
%          ^ markup.underline.underline.latex


% VERBATIM

\command{}
% ^ support.function.general.latex
\verb{\command{}}
%      ^ markup.raw.verb.latex
%      ^ meta.environment.verbatim.latex
%      ^ - support.function.general.latex
\verb|\command{}|
%      ^ markup.raw.verb.latex
%      ^ meta.environment.verbatim.latex
%      ^ - support.function.general.latex
\verb+\command{}+
%      ^ markup.raw.verb.latex
%      ^ meta.environment.verbatim.latex
%      ^ - support.function.general.latex

\begin{verbatim}
% ^ support.function.be.latex
%        ^ variable.parameter.function.latex
The \emph{verbatim} environment sets everything in verbatim.
% <- meta.environment.verbatim.latex
% ^ markup.raw.verbatim.latex
%         ^ - markup.italic.emph.latex
\command{}
% ^ - support.function.general.latex
% This is not a comment
% <- - comment.line.percentage.tex
\end{verbatim}


% COMMANDS INSIDE ARGUMENTS

\makebox[\linewidth]{...}
% ^ support.function.box.latex
%         ^ support.function.general.latex

\includegraphics[width=0.33\textwidth, angle=30]{test.png}
% ^ support.function.general.latex
%                           ^ support.function.general.latex


% PACKAGE: comment
% The comment package can be used to write block comment
% using an environment.

\begin{comment}
% ^ support.function.be.latex
%      ^ variable.parameter.function.latex
This environment can be used to write
% <- comment.block.environment.comment.latex
block comments.
% <- comment.block.environment.comment.latex
\end{comment}


\comment
% <- comment.block.command.comment.latex
% ^ punctuation.definition.comment.start.latex
This block comment can be done with
% <- comment.block.command.comment.latex
opening and closing commands.
% <- comment.block.command.comment.latex
\endcomment
% <- comment.block.command.comment.latex
% ^ punctuation.definition.comment.end.latex


\end{document}
% ^ support.function.be.latex
%        ^ variable.parameter.function.latex
